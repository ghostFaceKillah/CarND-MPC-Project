\documentclass[12pt]{article}
\usepackage{amsmath}

\begin{document}
\section{How MPC works?}

Controller gets reference path from path planner and it's job is to follow it.
In our case the path is represented as waypoints, so we transform it to slightly
more abstract representation by fitting a polynomial to it.


Additionaly, the controller has a model of behaviour of the car. Using this
model it can simulate what will happen with the car given current state and planned
actuations. Based on this, within a window of prediciton, the controller chooses
actions that minimize the cost. The cost is composed of many parts, penalizing various
aspects of controller behaviour:
\begin{itemize}
    \item Most importantly, the "distance" from reference wanted state:
    \begin{itemize}
        \item distance from center of the track
        \item distance from reference yaw
        \item distance from reference velocity
    \end{itemize}
    \item Usage of actuators
    \item Rate of change in usage of actuators
\end{itemize}

In order for controller to be successful, a balance has to be struck between
all of these constraints.


\section{Model}

The vehicle state contains 6 variables:
\begin{itemize}
    \item $x$, $y$ -  position coordinates
    \item $\psi$ - heading angle
    \item $v$ - velocity
    \item $cte$ - cross-track error
    \item $e\psi$ - heading angle error
\end{itemize}


Controlable actuators are:
\begin{itemize}
    \item $\delta$ - steering angle
    \item $a$ - acceleration
\end{itemize}

We update state based on simple kinematic model presented in the lectures:

\begin{align}
    x_{t+1} = x_t + v_t \cdot cos(\psi_t) \cdot dt \\
    y_{t+1} = y_t + v_t \cdot sin(\psi_t) \cdot dt \\
    \psi_{t+1} = \psi_t + \frac{v_t}{L_f} \cdot \delta_t \cdot dt \\
    v_{t+1} = v_t + a_t \cdot dt \\
    cte_{t+1} = f(x_t) - y_t + v_t \cdot sin(e\psi_t) \cdot dt \\
    e\psi_{t+1} = \psi_t - \hat \psi_t + \frac{v_t}{L_f} \cdot \delta_t \cdot dt
\end{align}

The only preprocessing I do is transforming the waypoints into the vehicle coordinate system.

\subsection{Choice of $N$ and $dt$}
Part of the project is choice of the parameters $N$ and $dt$. Following the sugestion in the
project statement, I have chosen $N = 10$ and $dt = 100 \text{ ms}$. Choosing $dt$ equal to
the expected latency allows me to simplify how I handle latency a lot.

\subsection{How to deal with latency?}
I have chosen a very simple way of dealing latency. While calculating state,
instead of using actuation values from one timestep before, I use actuations from two
timesteps before. Timestep in MPC is chosen to be equal to expected latency of 100 ms.


\end{document}
